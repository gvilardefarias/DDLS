\documentclass[
    12pt,
    a4paper,
    oneside,
    chapter=TITLE,
    section=TITLE,
    subsection=TITLE,
    subsubsection=TITLE,
    english,
    french,
    spanish,
    brazil,
    ]{abntex2}

\usepackage{lmodern}
\usepackage[T1]{fontenc}
\usepackage[utf8]{inputenc}
\usepackage{indentfirst}
\usepackage{color}
\usepackage{graphicx}
\usepackage{microtype}
\usepackage[brazilian,hyperpageref]{backref}
\usepackage[alf]{abntex2cite}

\usepackage{listings}

\renewcommand{\backrefpagesname}{Citado na(s) página(s):~}
\renewcommand{\backref}{}
\renewcommand*{\backrefalt}[4]{
    \ifcase #1
        Nenhuma citação no texto.
    \or
        Citado na página #2.
    \else
        Citado #1 vezes nas páginas #2.
    \fi}

\titulo{Chroma Key}
\autor{DDLS}
\local{Campina Grande}
\data{\today}
\instituicao{
  \textbf{Statement of Work}
  \par
  Instituto Federal de Educação, Ciência e Tecnologia da Paraíba
  }
\tipotrabalho{Projeto de Pesquisa}
\preambulo{Especificação de projeto feito com objetivo de apresentar e auxiliar o desenvolvimento do mesmo, apresentado à comunidade do IFPB}

\definecolor{blue}{RGB}{41,5,195}
\makeatletter
\hypersetup{
        %pagebackref=true,
        pdftitle={\@title},
        pdfauthor={\@author},
        pdfsubject={\imprimirpreambulo},
        pdfcreator={LaTeX with abnTeX2},
        pdfkeywords={abnt}{latex}{abntex}{abntex2}{projeto de pesquisa},
        colorlinks=false,
        linkcolor=blue,
        citecolor=blue,
        filecolor=magenta,
        urlcolor=blue,
        bookmarksdepth=4
}
\makeatother

\setlength{\parindent}{1.3cm}
\setlength{\parskip}{0.2cm}

\makeindex

\lstset{
  language=,
  basicstyle=\ttfamily\small,
  keywordstyle=\color{blue},
  stringstyle=\color{verde},
  commentstyle=\color{red},
  extendedchars=true,
  showspaces=false,
  showstringspaces=false,
  numbers=left,
  numberstyle=\tiny,
  breaklines=true,
  breakautoindent=true,
  captionpos=b,
  xleftmargin=0pt,
}

\begin{document}

\selectlanguage{brazil}
\frenchspacing 
\imprimircapa
\imprimirfolhaderosto

\pdfbookmark[0]{\contentsname}{toc}
\tableofcontents*
\cleardoublepage

\textual

\chapter{Problema}
Chroma Key é um tecnica de efeito visual que permite colocar uma imagem ou video sobre outro utilizando o anulamento de uma cor em umas das imagens, que no nosso caso será o verde.

\par
Implementação 1:
\begin{lstlisting}

\end{lstlisting}

\chapter{Interface}
A interface deverá seguir o padrão VGA e será uma comunicação assíncrona

\chapter{Estimulos}
\section{Entrada}
A entrada consistirá em 2 canais VGAs

\section{Saída}
A saida será um canal VGA com a imagem ou video já processado em tempo real

\section{Caso Base}
Na \autoref{fig-caso-PROJETO} temos um exemplo de como deve seu IP deve responder com os respectivos estímulos.


\end{document}
