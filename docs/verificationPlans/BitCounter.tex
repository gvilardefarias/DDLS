\documentclass[
    12pt,
    a4paper,
    oneside,
    chapter=TITLE,
    section=TITLE,
    subsection=TITLE,
    subsubsection=TITLE,
    english,
    french,
    spanish,
    brazil,
    ]{abntex2}

\usepackage{lmodern}
\usepackage[T1]{fontenc}
\usepackage[utf8]{inputenc}
\usepackage{indentfirst}
\usepackage{color}
\usepackage{graphicx}
\usepackage{microtype}
\usepackage[brazilian,hyperpageref]{backref}
\usepackage[alf]{abntex2cite}

\usepackage{hyperref}

\renewcommand{\backrefpagesname}{Citado na(s) página(s):~}
\renewcommand{\backref}{}
\renewcommand*{\backrefalt}[4]{
    \ifcase #1
        Nenhuma citação no texto.
    \or
        Citado na página #2.
    \else
        Citado #1 vezes nas páginas #2.
    \fi}

\titulo{BIT COUNTER}
\autor{DDLS}
\local{Campina Grande}
\data{\today}
\instituicao{
  \textbf{Statement of Work}
  \par
  Instituto Federal de Educação, Ciência e Tecnologia da Paraíba
  }
\tipotrabalho{Projeto de Pesquisa}
\preambulo{Plano de verificação feito com objetivo de auxiliar o desenvolvimento do testbench, apresentado à comunidade do IFPB}

\definecolor{blue}{RGB}{41,5,195}
\makeatletter
\hypersetup{
        %pagebackref=true,
        pdftitle={\@title}, 
        pdfauthor={\@author},
        pdfsubject={\imprimirpreambulo},
        pdfcreator={LaTeX with abnTeX2},
        pdfkeywords={abnt}{latex}{abntex}{abntex2}{projeto de pesquisa}, 
        colorlinks=false,               
        linkcolor=blue,             
        citecolor=blue,             
        filecolor=magenta,          
        urlcolor=blue,
        bookmarksdepth=4
}
\makeatother

\setlength{\parindent}{1.3cm}
\setlength{\parskip}{0.2cm}

\makeindex

\begin{document}

\selectlanguage{brazil}
\frenchspacing 
\imprimircapa
\imprimirfolhaderosto

\pdfbookmark[0]{\contentsname}{toc}
\tableofcontents*
\cleardoublepage

\textual

\chapter{Resumo do Sistema}
\href{run:../specifications/BitCounter.pdf}{Link para especificação do sistema}

\chapter{Modelo de Referência}
O modelo de referência por ser extrememente simples foi desenvolvido pela equipe e não se baseia em nenhum padrão, fazendo exatamente o que se necessita.

\chapter{Definição da Transação}
A transação será sequencias de 32 bits quaisquer, já que não se tem nenhum padrão, apenas deve ser tentado o equilibrio entre o valor do primeiro bit.

\chapter{Definição da Cobertura Funcional e dos Estímulos}
\section{Direcionados:}
Os casos direcionados testarão o básico do DUT\footnote{Device Under Test}, será testado a contagem de bits 1s e a de 0s com as seguintes palavras binarias: "10000000000000000000000000000001" e "01111111111111111111111111111110".

\section{Casos limite:}
Casos limite são a parte crucial dos testes, pois eles que são responsáveis por levar o DUT a situações extremas. Com isso podemos testar o máximo e o mínimo de cada contagem, utilizando as seguintes palavras: "10000000000000000000000000000000", "11111111111111111111111111111111", "01111111111111111111111111111111" e "00000000000000000000000000000000".

\section{Pseudo-Aleatórios: direcionado}
Para garantir a consistência do sistema será mandada 500 palavras aleatórios, sendo 250 começando com o bit 1 e os outros 250 começando com o bit 0.

\end{document}